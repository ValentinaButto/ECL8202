% Options for packages loaded elsewhere
\PassOptionsToPackage{unicode}{hyperref}
\PassOptionsToPackage{hyphens}{url}
%
\documentclass[
]{article}
\usepackage{amsmath,amssymb}
\usepackage{iftex}
\ifPDFTeX
  \usepackage[T1]{fontenc}
  \usepackage[utf8]{inputenc}
  \usepackage{textcomp} % provide euro and other symbols
\else % if luatex or xetex
  \usepackage{unicode-math} % this also loads fontspec
  \defaultfontfeatures{Scale=MatchLowercase}
  \defaultfontfeatures[\rmfamily]{Ligatures=TeX,Scale=1}
\fi
\usepackage{lmodern}
\ifPDFTeX\else
  % xetex/luatex font selection
\fi
% Use upquote if available, for straight quotes in verbatim environments
\IfFileExists{upquote.sty}{\usepackage{upquote}}{}
\IfFileExists{microtype.sty}{% use microtype if available
  \usepackage[]{microtype}
  \UseMicrotypeSet[protrusion]{basicmath} % disable protrusion for tt fonts
}{}
\makeatletter
\@ifundefined{KOMAClassName}{% if non-KOMA class
  \IfFileExists{parskip.sty}{%
    \usepackage{parskip}
  }{% else
    \setlength{\parindent}{0pt}
    \setlength{\parskip}{6pt plus 2pt minus 1pt}}
}{% if KOMA class
  \KOMAoptions{parskip=half}}
\makeatother
\usepackage{xcolor}
\usepackage[margin=1in]{geometry}
\usepackage{color}
\usepackage{fancyvrb}
\newcommand{\VerbBar}{|}
\newcommand{\VERB}{\Verb[commandchars=\\\{\}]}
\DefineVerbatimEnvironment{Highlighting}{Verbatim}{commandchars=\\\{\}}
% Add ',fontsize=\small' for more characters per line
\usepackage{framed}
\definecolor{shadecolor}{RGB}{248,248,248}
\newenvironment{Shaded}{\begin{snugshade}}{\end{snugshade}}
\newcommand{\AlertTok}[1]{\textcolor[rgb]{0.94,0.16,0.16}{#1}}
\newcommand{\AnnotationTok}[1]{\textcolor[rgb]{0.56,0.35,0.01}{\textbf{\textit{#1}}}}
\newcommand{\AttributeTok}[1]{\textcolor[rgb]{0.13,0.29,0.53}{#1}}
\newcommand{\BaseNTok}[1]{\textcolor[rgb]{0.00,0.00,0.81}{#1}}
\newcommand{\BuiltInTok}[1]{#1}
\newcommand{\CharTok}[1]{\textcolor[rgb]{0.31,0.60,0.02}{#1}}
\newcommand{\CommentTok}[1]{\textcolor[rgb]{0.56,0.35,0.01}{\textit{#1}}}
\newcommand{\CommentVarTok}[1]{\textcolor[rgb]{0.56,0.35,0.01}{\textbf{\textit{#1}}}}
\newcommand{\ConstantTok}[1]{\textcolor[rgb]{0.56,0.35,0.01}{#1}}
\newcommand{\ControlFlowTok}[1]{\textcolor[rgb]{0.13,0.29,0.53}{\textbf{#1}}}
\newcommand{\DataTypeTok}[1]{\textcolor[rgb]{0.13,0.29,0.53}{#1}}
\newcommand{\DecValTok}[1]{\textcolor[rgb]{0.00,0.00,0.81}{#1}}
\newcommand{\DocumentationTok}[1]{\textcolor[rgb]{0.56,0.35,0.01}{\textbf{\textit{#1}}}}
\newcommand{\ErrorTok}[1]{\textcolor[rgb]{0.64,0.00,0.00}{\textbf{#1}}}
\newcommand{\ExtensionTok}[1]{#1}
\newcommand{\FloatTok}[1]{\textcolor[rgb]{0.00,0.00,0.81}{#1}}
\newcommand{\FunctionTok}[1]{\textcolor[rgb]{0.13,0.29,0.53}{\textbf{#1}}}
\newcommand{\ImportTok}[1]{#1}
\newcommand{\InformationTok}[1]{\textcolor[rgb]{0.56,0.35,0.01}{\textbf{\textit{#1}}}}
\newcommand{\KeywordTok}[1]{\textcolor[rgb]{0.13,0.29,0.53}{\textbf{#1}}}
\newcommand{\NormalTok}[1]{#1}
\newcommand{\OperatorTok}[1]{\textcolor[rgb]{0.81,0.36,0.00}{\textbf{#1}}}
\newcommand{\OtherTok}[1]{\textcolor[rgb]{0.56,0.35,0.01}{#1}}
\newcommand{\PreprocessorTok}[1]{\textcolor[rgb]{0.56,0.35,0.01}{\textit{#1}}}
\newcommand{\RegionMarkerTok}[1]{#1}
\newcommand{\SpecialCharTok}[1]{\textcolor[rgb]{0.81,0.36,0.00}{\textbf{#1}}}
\newcommand{\SpecialStringTok}[1]{\textcolor[rgb]{0.31,0.60,0.02}{#1}}
\newcommand{\StringTok}[1]{\textcolor[rgb]{0.31,0.60,0.02}{#1}}
\newcommand{\VariableTok}[1]{\textcolor[rgb]{0.00,0.00,0.00}{#1}}
\newcommand{\VerbatimStringTok}[1]{\textcolor[rgb]{0.31,0.60,0.02}{#1}}
\newcommand{\WarningTok}[1]{\textcolor[rgb]{0.56,0.35,0.01}{\textbf{\textit{#1}}}}
\usepackage{graphicx}
\makeatletter
\def\maxwidth{\ifdim\Gin@nat@width>\linewidth\linewidth\else\Gin@nat@width\fi}
\def\maxheight{\ifdim\Gin@nat@height>\textheight\textheight\else\Gin@nat@height\fi}
\makeatother
% Scale images if necessary, so that they will not overflow the page
% margins by default, and it is still possible to overwrite the defaults
% using explicit options in \includegraphics[width, height, ...]{}
\setkeys{Gin}{width=\maxwidth,height=\maxheight,keepaspectratio}
% Set default figure placement to htbp
\makeatletter
\def\fps@figure{htbp}
\makeatother
\setlength{\emergencystretch}{3em} % prevent overfull lines
\providecommand{\tightlist}{%
  \setlength{\itemsep}{0pt}\setlength{\parskip}{0pt}}
\setcounter{secnumdepth}{-\maxdimen} % remove section numbering
\ifLuaTeX
  \usepackage{selnolig}  % disable illegal ligatures
\fi
\IfFileExists{bookmark.sty}{\usepackage{bookmark}}{\usepackage{hyperref}}
\IfFileExists{xurl.sty}{\usepackage{xurl}}{} % add URL line breaks if available
\urlstyle{same}
\hypersetup{
  pdftitle={Régression robuste aux valeurs extrêmes},
  hidelinks,
  pdfcreator={LaTeX via pandoc}}

\title{Régression robuste aux valeurs extrêmes}
\author{}
\date{\vspace{-2.5em}}

\begin{document}
\maketitle

Ce travail doit être remis avant le \textbf{23 février avant 22h} sur
Moodle.

\hypertarget{donnuxe9es}{%
\subsection{Données}\label{donnuxe9es}}

Cet exercice est basé sur le jeu de données \emph{gapminder} du package
du même nom.

\begin{quote}
Jennifer Bryan (2017). gapminder: Data from Gapminder. R package version
0.3.0. \url{https://CRAN.R-project.org/package=gapminder}
\end{quote}

Ce jeu de données inclut l'espérance de vie (\emph{lifeExp}), la
population (\emph{pop)} et le PIB par habitant (\emph{gdpPercap}) pour
142 pays et 12 années (aux 5 ans entre 1952 et 2007).

\begin{Shaded}
\begin{Highlighting}[]
\FunctionTok{library}\NormalTok{(gapminder)}
\FunctionTok{str}\NormalTok{(gapminder)}
\end{Highlighting}
\end{Shaded}

\begin{verbatim}
## tibble [1,704 x 6] (S3: tbl_df/tbl/data.frame)
##  $ country  : Factor w/ 142 levels "Afghanistan",..: 1 1 1 1 1 1 1 1 1 1 ...
##  $ continent: Factor w/ 5 levels "Africa","Americas",..: 3 3 3 3 3 3 3 3 3 3 ...
##  $ year     : int [1:1704] 1952 1957 1962 1967 1972 1977 1982 1987 1992 1997 ...
##  $ lifeExp  : num [1:1704] 28.8 30.3 32 34 36.1 ...
##  $ pop      : int [1:1704] 8425333 9240934 10267083 11537966 13079460 14880372 12881816 13867957 16317921 22227415 ...
##  $ gdpPercap: num [1:1704] 779 821 853 836 740 ...
\end{verbatim}

\hypertarget{effet-du-pib-et-du-temps-sur-lespuxe9rance-de-vie}{%
\subsection{1. Effet du PIB et du temps sur l'espérance de
vie}\label{effet-du-pib-et-du-temps-sur-lespuxe9rance-de-vie}}

\begin{enumerate}
\def\labelenumi{\alph{enumi})}
\tightlist
\item
  Visualisez d'abord les données d'espérance de vie en fonction du PIB
  par habitant et de l'année. Il est suggéré de représenter le
  logarithme de \emph{gdpPercap} et de séparer les différentes années,
  par exemple avec des facettes dans \emph{ggplot2}:
  \texttt{...\ +\ facet\_wrap(\textasciitilde{}year)}.
\end{enumerate}

Quelles tendances générales observez-vous? Semble-t-il y avoir des
valeurs extrêmes qui pourraient influencer fortement un modèle de
régression? Si oui, essayez d'identifier ces données dans le tableau en
vous basant sur la position des points dans le graphique.

\begin{enumerate}
\def\labelenumi{\alph{enumi})}
\setcounter{enumi}{1}
\tightlist
\item
  Réalisez une régression linéaire (\texttt{lm}) pour déterminer l'effet
  du PIB par habitant, de l'année et de leur interaction sur l'espérance
  de vie. Pour faciliter l'inteprétation des coefficients, effectuez les
  transformations suivantes sur les prédicteurs:
\end{enumerate}

\begin{itemize}
\item
  Prenez le logarithme de \emph{gdpPercap} et normalisez-le avec la
  fonction \texttt{scale}. \emph{Rappel}: \texttt{scale(x)} soustrait de
  chaque valeur de \texttt{x} leur moyenne et divise par leur
  écart-type, donc la variable résultante a une moyenne de 0 et un
  écart-type de 1; elle représente le nombre d'écarts-types au-dessus ou
  en-dessous de la moyenne.
\item
  Remplacez \emph{year} par le nombre d'années écoulées depuis 1952.
\end{itemize}

Interprétez la signification de chacun des coefficients du modèle, puis
consultez les graphiques de diagnostic. Les suppositions du modèle
linéaire sont-elles respectées?

\begin{enumerate}
\def\labelenumi{\alph{enumi})}
\setcounter{enumi}{2}
\tightlist
\item
  Comparez le résultat du modèle en (b) avec deux alternatives plus
  robustes aux valeurs extrêmes: la régression robuste basée sur le
  bipoids de Tukey (fonction \texttt{lmrob} du package
  \emph{robustbase}) et la régression de la médiane (fonction
  \texttt{rq} du package \emph{quantreg}, en choisissant seulement le
  quantile médian). Expliquez comment les estimés des coefficients et
  leurs erreurs-types diffèrent entre les trois méthodes.
\end{enumerate}

\emph{Note}: Utilisez l'option \texttt{showAlgo\ =\ FALSE} en appliquant
la fonction \texttt{summary} au résultat de \texttt{lmrob}, pour
simplifier le sommaire.

\begin{enumerate}
\def\labelenumi{\alph{enumi})}
\setcounter{enumi}{3}
\tightlist
\item
  Superposez les droites de régression des trois modèles sur le
  graphique en (a). Avec \texttt{ggplot}, vous pouvez utiliser la
  fonction \texttt{geom\_smooth} avec \texttt{method\ =\ "lm"} pour la
  régression linéaire et \texttt{method\ =\ "lmrob"} pour la régression
  robuste. Pour la régression de la médiane, vous pouvez utiliser
  \texttt{geom\_quantile} tel que vu dans les notes.
\end{enumerate}

\hypertarget{variation-des-effets-par-quantile}{%
\subsection{2. Variation des effets par
quantile}\label{variation-des-effets-par-quantile}}

\begin{enumerate}
\def\labelenumi{\alph{enumi})}
\item
  D'après votre observation des données en 1(a), serait-il utile de
  modéliser différents quantiles de l'espérance de vie en fonction des
  prédicteurs? Justifiez votre réponse.
\item
  Réalisez une régression quantile avec les mêmes prédicteurs qu'en
  1(b), avec les quantiles suivants:
  \texttt{(0.1,\ 0.25,\ 0.5,\ 0.75,\ 0.9)}. Utilisez la fonction
  \texttt{plot} sur le sommaire de la régression quantile et décrivez
  comment l'effet des prédicteurs varie entre les quantiles.
\item
  Superposez les droites de régression quantile au graphique des
  données. Les tendances pour chaque quantile semblent-elles affectées
  par des valeurs extrêmes?
\end{enumerate}

\hypertarget{note-sur-les-comparaisons-internationales}{%
\subsection{Note sur les comparaisons
internationales}\label{note-sur-les-comparaisons-internationales}}

Bien que ce jeu de données soit utile pour illustrer les concepts de
régression robuste et de régression quantile, soulignons que ce type
d'analyse statistique comparant des variables mesurées au niveau
national comporte plusieurs limites:

\begin{itemize}
\item
  On ne peut pas supposer que les associations détectées s'appliquent à
  une échelle plus petite (ex.: le lien entre espérance de vie et revenu
  en comparant les moyennes nationales n'est pas nécessairement le même
  que le lien entre espérance de vie et revenu au niveau des individus
  habitant chaque pays).
\item
  Les moyennes calculées dans différents pays ne sont pas des
  observations indépendantes, car les conditions environnementales,
  sociales et économiques sont corrélées entre pays proches.
\item
  Il y a de nombreux facteurs qui différencient les pays, donc il est
  difficile d'interpréter une association comme un lien de cause à
  effet.
\end{itemize}

Beaucoup d'articles, en particulier dans les domaines des sciences
sociales, ont été publiés au sujet des méthodes à suivre pour réaliser
ce type de comparaisons internationales (\emph{cross-country
comparisons}).

\end{document}
